
% Considering that the solubility of H2O and CO2 in silicate melts is mainly dependent on pressure, the concentrations of H2O and CO2 determined in glass inclusions and volcanic gasses are commonly used to estimate magma storage conditions and magma evolution history, provided that the volatile solubility data and models are available.

% Water is present in magmas in higher concentrations than CO2 and is also more soluble at lower pressures, and, therefore it is the dominant volatile forming bubbles during volcanic eruptions. Dissolved water affects both phase equilibria and melt physical properties such as density and viscosity, therefore, water is important for understanding magmatic processes. 

% The amount of water dissolved in magmas at depth beneath volcanoes is fundamental to a wide range of magmatic and eruptive processes due to water’s dominant control on magma generation, viscosity, and buoyancy. 

% The concentration of H2O in magmas has been an elusive but critical parameter to the understanding of several fundamental processes, including subduction zone recycling, mantle melting, magma differentiation, and magma ascent and eruption (Cashman 2004). Because the solubility of water in silicate melts decreases substantially at low pressures, H2O-rich magmas will degas upon ascent, erupting with a minute fraction of their original dissolved concentration.

% This study utilizes the strong pressure dependence on the solubility of CO2 and H2O in silicate melts to determine the pressures at which pockets of melt, also known as melt inclusions, were trapped within olivine crystals. Through prior constraints on the density profile of the crust, entrapment pressures from F8 melt inclusions erupted in late May, mid-July and early August 2018 can be converted into entrapment depths. 